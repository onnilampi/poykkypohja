% latex-minutes.tex
\documentclass[finnish,12pt,a4paper]{article}
\pagestyle{plain}

\usepackage{graphicx}
\usepackage{url}
\usepackage{pstricks}
\usepackage{eurosym}
\usepackage{tikz}
\usepackage{verbatim}
\usepackage{wrapfig}
\usepackage[T1]{fontenc}
\usepackage{lmodern}
\usepackage[colorlinks=true,
    linkcolor=blue,
    urlcolor=blue,
    citecolor=blue,
    anchorcolor=blue,
bookmarks=true]{hyperref}
\usepackage{float}
\usepackage[british]{babel}
\usepackage[nodayofweek]{datetime}
\usepackage[utf8]{inputenc}
\setlength{\topmargin}{0in}
\setlength{\headheight}{0in}
\setlength{\headsep}{0in}
\setlength{\textheight}{10.0in}
\setlength{\footskip}{18pt}
\setlength{\oddsidemargin}{0in}
\setlength{\textwidth}{6in}
\setlength{\marginparwidth}{1in}
\renewcommand{\familydefault}{\sfdefault}
\let\margin\marginpar
\newcommand\myMargin[1]{\margin{\raggedright\scriptsize #1}}
\renewcommand{\marginpar}[1]{\myMargin{#1}}
%omia makroja
\newcommand{\ekatarkastaja}{Penis Pippeli}
\newcommand{\tokatarkastaja}{Idi Juutti}
\newcommand{\ekalaskija}{Eka Laskija}
\newcommand{\tokalaskija}{Toka Laskija}
\newcommand{\puhis}{Pertsa Puhaanjohtaja}
\newcommand{\sihteeri}{Sami Sihteeri}
\newcommand{\alkaa}{00:00}
\newcommand{\loppuu}{00:01}
\newcommand{\aika}{1.1.1970}
\newcommand{\paikka}{Paikka X}
\newcommand{\action}[3]{\marginpar{#1 [#2]}}
%näitä lisää tarpeen mukaan
\newcommand{\ekaliite}[1]{\newpage \section*{Liite 1 - #1}}

\begin{document}

\title{Pöytäkirja}
\author{Yhdistys}
\date{\aika{}}
\maketitle


\begin {center}
\textbf{Paikka:} \paikka{}
\end{center}
\textbf{Hallituksesta paikalla:} 
\begin{itemize}
    \item Ensimmäinen Ukki (<virka>)
    \item Toinen Ukki (myöhässä, saapui <saapumisaika>)
    \item jne...
\end{itemize}
\textbf{Poissa:} 
\begin{itemize}
    \item Ensimmäinen Ukki
    \item Toinen Ukki
    \item jne...
\end{itemize}
\textbf{Hallituksen ulkopuolelta:} 
\begin{itemize}
    \item Ensimmäinen Ukki
    \item Toinen Ukki
    \item jne...
\end{itemize}

\section{Kokouksen avaus}
\begin{itemize}
    \item Puheenjohtaja avasi kokouksen kello \alkaa{}.
\end{itemize}

\section{Kokouksen järjestäytyminen}
\begin{itemize}
    \item Valittiin kokouksen puheenjohtajaksi \puhis{} ja sihteeriksi \sihteeri{}.
    \item Valittiin pöytäkirjantarkastajiksi \ekatarkastaja{} ja \tokatarkastaja{}.
        %% Tarvittaessa
        %\item Valittiin ääntenlaskijoiksi \ekalaskija{} ja \tokalaskija{}.
\end{itemize}

\section{Todetaan kokouksen laillisuus ja päätösvaltaisuus}
\begin{itemize}
    \item Todettiin laillisuus ja päätösvaltaisuus, korruptiokulut 20\euro{}.
\end{itemize}

\section{Hyväksytään kokouksen työjärjestys}
\begin{itemize}
    \item Ei muutoksia esityslistaan, hyväksytään se suoraan kokouksen työjärjestykseksi.
\end{itemize}

\section{Ilmoitusasiat}
\begin{itemize}
    \item Kokoustarjoiluja on.
\end{itemize}
\subsection{Kuulumiskierros}
\begin{itemize}
    \item asdasd
    \item qweqwe
\end{itemize}

\section{Muut esille tulevat asiat}
\begin{itemize}
    \item Ei muita esille tulevia asioita
\end{itemize}
\section{Kokouksen päättäminen}
\begin{itemize}
    \item Puheenjohtaja päätti kokouksen ajassa \loppuu{}. 
\end{itemize}

\noindent\begin{tabular}{ll}
    \\[8ex]
    \makebox[2.5in]{\hrulefill} & \makebox[2.5in]{\hrulefill}\\
    \puhis{}, puheenjohtaja & \sihteeri{}, sihteeri\\[4ex]% adds space between the two sets of signatures
    \\[4ex]
    \makebox[2.5in]{\hrulefill} & \makebox[2.5in]{\hrulefill}\\
    \ekatarkastaja{}, pöytäkirjan tarkastaja& \tokatarkastaja{}, pöytäkirjan tarkastaja\\[8ex]% 
\end{tabular}

\ekaliite{Liitteen nimi}
<Liitten sisältö>
\end{document}
